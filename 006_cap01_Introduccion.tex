
\chapter{INTRODUCCION}
\newpage

En el ambiente de la tecnolog\'ia y principalmente en el ambiente del desarrollo de software los cambios y actualizaciones son constantes, es de la misma forma que este proyecto propone un nuevo enfoque a los sistemas de gesti\'on de contenidos o CMS's.\\
Los CMS's en la actualidad siguen patrones bien definidos para la creaci\'on de contenidos, muchas veces estos procesos requieren de una formaci\'on espec\'ifica lo cual dificulta la utilizaci\'on de los mismos. La aplicaci\'on iCMS desarrollada durante este proyecto se presenta como una alternativa frente a los CMS's tradicionales (Joomla, Drupal, Zikula, etc.), iCMS es un CMS de f\'acil utilizaci\'on tanto para desarrolladores como para usuarios finales, iCMS logra esta cualidad gracias a que muestra gran parte de los controles para la creaci\'on del contenido desde la misma pantalla inicial.\\
Otra de las caracter\'isticas principales de iCMS es la herramienta de personalizaci\'on de la plantilla, este trabaja sobre la plantilla base, permitiendo crear nuevas secciones para el contenido, definir o cambiar la secci\'on del contenido principal, entre otras.\\
Todos los CMS's son dise\~nados para cumplir con un objetivo bien marcado (portales web, e-commerce, e-learning, etc.) e iCMS no es la excepci\'on, iCMS ha sido definida para la creaci\'on de portales web, aunque con la agregaci\'on de extensiones (modulos, bloques, temas) la funcionalidad b\'asica puede ser ampliada enormemente.

%\end{document}
\clearpage
