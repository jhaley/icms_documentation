\chapter*{Resumen}
\addcontentsline{toc}{chapter}{Resumen}

\thispagestyle{empty}

%%% An approach to writing a thesis abstract
%%% ========================================
%%% Begin by identifying in a sentence the main purpose of the thesis.
%%%
%%% Then write answers to each of these questions:
%%%
%%% 1.- What is the problem or question that the work addresses?
%%% 2.- Why is it important?
%%% 3.- How was the investigation undertaken?
%%% 4.- What was found and what does it mean?
%%%
%%% You should find the answers to questions 1 and 2 in your Introduction, ProblemSpecification and SolveSpecification; 
%%% the answer to question 3 will be a summary of your Methods; 
%%% and the answer to question 4 will summarise your Results, Discussion and Conclusion.

El desarrollo de un Sistema de Gesti\'on de Contenidos es un proyecto grande, ya que deben ser implementados distintos m\'odulos para dar soporte a las crecientes demandas de los usuarios, as\'i como tambi\'en las demandas de los desarrolladores.\\
\newline
Los Sistemas de Gesti\'on de Contenidos en la actualidad tienen procesos y configuraciones espec\'ificos para la manipulaci\'on de los contenidos (art\'iculos, secciones, noticias, men\'us, etc.). Por otro lado, la 
gesti\'on de las plantillas y principalmente su personalizaci\'on es un proceso que requiere de mucho tiempo y dedicaci\'on.\\
\newline
Estos dos caracter\'isticas hacen que se consuma mucho tiempo de desarrollo en procesos que pueden ser simplificados o incluso automatizados. En el presente proyecto se ha desarrollado un CMS que cumple con los siguientes puntos:
\begin{itemize}
	\item[-] Las configuraciones se hacen en su mayor\'ia de forma autom\'atica.
	\item[-] La personalizaci\'on de las plantillas son realizadas directamente desde el CMS de forma gr\'afica.
	\item[-] El contenido es personalizado directamente sobre la plantilla.
	\item[-] Por el uso de librer\'ias personalizadas \textit{i}CMS tiene un tiempo de respuesta menor en comparaci\'on con otros CMSs.
	\item[-] Soporte para la renderizaci\'on extensiones (M\'odulos, Bloques y Plantillas).
	\item[-] Soporte para la instalaci\'on de nuevas plantillas.
\end{itemize}
\textit{i}CMS tiene una arquitectura basada en un motor de renderizado de plantillas como elemento base, adicionalmente se sirve de un conjunto de renderizadores especializados para las dem\'as extensiones (M\'odulos y Bloques) soportados por \textit{i}CMS.\\
\newline
Cada m\'odulo sigue el patr\'on MVC (Modelo-Vista-Controlador), esto es muy \'util para tener un renderizador est\'andar para todos los m\'odulos.\\
\newline
Para la mayor parte del \textit{i}CMS se ha utilizado las tecnolog\'ias AJAX y JavaScript, con lo cual se ha obtenido gran dinamicidad en la personalizaci\'on tanto de la plantilla como del contenido.\\
\newline
En conclusi\'on, se encontr\'o  que el uso de tecnolog\'ias como AJAX, JavaScript, JSON y XML combinado con librer\'ias personalizadas hacen que se pueda mejorar la experiencia del usuario.\\

\clearpage
