\part{DESARROLLO DEL PROYECTO}

\chapter{\textit{i}CMS}
\newpage
\section{Introducci\'on}
El proyecto iCMS no sigue una metodolog\'ia de desarrollo en particular, por lo tanto este cap\'itulo muestra los elementos utilizados tanto para el desarrollo mismo del proyecto como tambi\'en parte de los resultados obtenidos (modelos).\\

El desarrollo del proyecto fue basado en una lista de tareas para poder conseguir la arquitectura inicial del \textit{i}CMS.\\

\section{Lista de tareas}
La lista de tareas que se muestran a continuaci\'on sirvio de base para poder desarrollar el \textit{i}CMS.

\begin{itemize}
\item Definir estructura de Directorios y Archivos.
\item Definir configuraciones b\'asicas.
\item Desarrollar renderizador para las plantillas.
\item Dise\~nar estructura de los m\'odulos y bloques.
\item Desarrollar renderizador para m\'odulos y bloques.
\item Desarrollar clases Singleton (Request, DBO, Factory, Editor WYSIWYG).
\item Desarrollar bloque de menus.
\item Desarrollar bloque de usuarios (pantalla de acceso - login)
\item Desarrollar modulo de contenido.
\item Desarrollar modulo de usuarios (formulario de registro, modificación de datos).
\item Crear librerías de “drag and drop” personalizadas y generalizadas para cualquier tipo de contenido (artículos, bloques, etc.).
\end{itemize}

Durante el desarrollo de estas tareas, han ido surgiendo nuevas tareas m\'as especializadas.

\begin{itemize}
\item Desarrollar m\'odulo de men\'us.
\item Desarrollar bloque de navegaci\'on (breadcrumb).
\item Desarrollar bloque y m\'odulo de noticias.
\item Mejorar el renderizador de plantillas para soportar el reordenamiento de bloques.
\item Actualizar el renderizador de plantillas para soportar soportar la edici\'on de la plantilla.
\end{itemize}

\section{Arquiterctura}
Para dise\~nar la arquitectura del sistema \textit{i}CMS se considerar\'on dos patrones arquitect\'onicos, el patr\'on MVC (Modelo-Vista-Controlador) y el patr\'on MVP (Modelo-Vista-Presentador).\\

Mucho de este patr\'on se ven en los distintos m\'odulos que componen el \textit{i}CMS, tambi\'en vemos otros patrones presentes en el desarrollo del proyecto (Singleton, Factory Method).\\

A continuaci\'on se presenta un peque\~no ejemplo sin el patr\'on MVC y luego el mismo ejemplo pero aplicando el patr\'on MVC y el patr\'on MVP:\\

Como se puede apreciar en el siguiente ejemplo, todos los elementos de la aplicaci\'on estan mezclados en un solo archivo.

\begin{lstlisting}[label=mvc_all_in_one,caption=Ejemplo sin ning\'un patr\'on,language=PHP]
<?php 
$db = new PDO('mysql:host=localhost;dbname=bd', 'root', 'password');	
$consulta = $db->prepare('SELECT * FROM items');
$consulta->execute();
$items = $consulta->fetchAll();
?>
<div>
    <table class="adminlist" cellspacing="1">
    	<thead>
        	<tr>
				<th class="title">ID</th>
                <th nowrap="nowrap">Nombre</th>
                <th nowrap="nowrap">Descripcio^oacute<n</th>
            </tr>
        </thead>
        <tbody>
        	<?php foreach ($items as $row) {
        	$link = 'index.php?accion=alguna_accion&id='. $row->id;
        	?>
            <tr>
                <td><?php echo $row->id; ?></td>
                <td><a href="<?php echo $link; ?>"><?php echo $row->nombre; ?></a></td>
                <td><a href="<?php echo $link; ?>"><?php echo $row->descripcion; ?></a></td>
            </tr>
            <?php } ?>
        </tbody>
    </table>
</div>
\end{lstlisting}


El patr\'on MVC separa los elementos de una aplicacion en tres grupos (Modelos, Vistas y Controladores).\\

El controlador es el elemento que se encarga de gestionar todas los eventos que son generados por el usuario a trav\'es de las Vistas, as\'i como tambi\'en de generar las vistas que son el resultado de dichas acciones.

\begin{lstlisting}[label=mvc_controlador,caption=Clase Controlador,language=PHP]
<?php

class Controlador {
	
	public function mostrar(){
		$vista = new Vista();
		$vista->setPlantilla('principal');
		$vista->mostrar();
	}
}
?>
\end{lstlisting}


El modelo es el elemento encargado de la gesti\'on de los datos (almacenamiento y recuperaci\'on).

\begin{lstlisting}[label=mvc_modelo,caption=Clase Modelo,language=PHP]
<?php

class Modelo {
	protected $db;
	public function __construct() {
		$this->db = new PDO('mysql:host=localhost;dbname=bd', 'root', 'password');	
	}
	
	public function getDatos(){
		$consulta = $this->db->prepare('SELECT * FROM items');
		$consulta->execute();
		return $consulta->fetchAll();
	}
}
?>
\end{lstlisting}


La vista es el elemento encargado de estructurar los datos recuperados para presentarlos al usuario, la vista conoce la existencia del modelo y puede interactuar con el.

\begin{lstlisting}[label=mvc_vista,caption=Clase Vista,language=PHP]
<?php
class Vista {
	protected $layout;
    public function mostrar(){
    	$model = new Modelo();
    	$items = $model->getDatos();
    	require 'plantillas/' . $layout . '.php';
    }

	public function setLayout($_layout = ''){
    	$this->layout = $_layout;
    }
}
?>
\end{lstlisting}


La plantilla es un elemento auxiliar de la vista, no es una clase, sino un archivo html con el m\'inimo que sirve para maquetar los datos obtenidos en la vista.

\begin{lstlisting}[label=mvc_plantilla,caption=Plantilla de la vista,language=HTML]
<html>
	<head></head>
	<body>
		<div>
		    <table class="adminlist" cellspacing="1">
    			<thead>
		        	<tr>
						<th class="title">ID</th>
                		<th nowrap="nowrap">Nombre</th>
		                <th nowrap="nowrap">Descripcio^oacute<n</th>
        		    </tr>
		        </thead>
        		<tbody>
		        	<?php foreach ($items as $row) {
        			$link = 'index.php?accion=alguna_accion&id='. $row->id;
		        	?>
        		    <tr>
                		<td><?php echo $row->id; ?></td>
		                <td><a href="<?php echo $link; ?>"><?php echo $row->nombre; ?></a></td>
        		        <td><a href="<?php echo $link; ?>"><?php echo $row->descripcion; ?></a></td>
		            </tr>
        		    <?php } ?>
		        </tbody>
		    </table>
		</div>
	</body>
</html>
\end{lstlisting}


El patr\'on MVP separa (al igual que el patr\'on MVC) los elementos de una aplicaci\'on en tres grupos (Modelos, Vistas y Presentadores). La mayor diferencia entre MVC y MVP es que en el patr\'on MVP, la Vista no sabe de la existencia del Modelo, algo que no sucede en el patr\'on MVC.\\

El presentador es b\'asicamente un controlador, pero no delega la responsabilidad de procesar la informaci\'on que llega ya sea desde la vista o del modelo.

\begin{lstlisting}[label=mvp_presentador,caption=Clase Presentador,language=PHP]
<?php
class Presentador {
	public function mostrar(){
		$modelo = new Modelo();
		$items = $modelo->getDatos();
		require 'principal.php';
	}
}
?>
\end{lstlisting}


El modelo trabaja de la misma forma que en el patr\'on MVC.

\input{codigos/007_ejemplo_ejemplo007}

La vista es el elemento encargado de estructurar los datos recuperados por el presentador para mostrarlos al usuario.

\input{codigos/008_ejemplo_ejemplo008}

Dadas las enormes similitudes entre estos dos patrones, se decidi\'o hacer uso del patr\'on MVC, la raz\'on para esta elecci\'on es que todas las extensiones tienen caracteristicas similares pero no iguales a las de Joomla, con lo cual se plantea para un futuro brindar el soporte necesario para que \textit{i}CMS sea totalmente compatible con las extensiones de Joomla.\\

\section{Modelo de Clases}
Como ya fue citado en el marco referencial, el modelo clases es la herramienta que da soporte, de alguna manera a la Programaci\'on Orientada a Objetos. Representando a los objetos que son participes del de sistema \textit{i}CMS.\\
A continuaci\'on podemos ver el modelo de clases general del proyecto \textit{i}CMS.

\newpage
\begin{figure}[h]
\centering
\includegraphics[scale=.24, keepaspectratio=true]{imagenes/03_imagen.png}
\caption{Modelo de clases general. [Elaboraci\'on propia].}
\end{figure}
\clearpage
\newpage

\subsection{Aplicaci\'on de los patrones \textit{Singleton} y \textit{Factory Method}}
La aplicaci\'on de estos patrones en el sistema se da en tres clases que son utlizadas por todo el sistema, desde el \textit{Framework Jhaley} hasta las extensiones (m\'odulos y bloques).\\

\begin{figure}[h]
\centering
\includegraphics[scale=.5, keepaspectratio=true]{imagenes/04_imagen.png}
\caption{Modelo de clases: Patrones Singleton y Factory Method. [Elaboraci\'on propia].}
\end{figure}

La clase \emph{JhaFactory} es la encargada de crear las instancias Singleton, si la instancia ya existe, entonces esta es reutilizada.

\begin{lstlisting}[label=factory_method,caption=Clase Factory Method]
<?php 
defined( '_JHAEXEC' ) or die( 'Access Denied' );
jhaimport('jhaley.base.object');

class JhaFactory extends JhaObject {
    static function &getRenderer($type = null){
        static $renderer;
       	if($type == null){
       		if(JhaRequest::getVar('elem') == 'mod_base' && JhaRequest::getVar('controller') == 'theme' && (JhaRequest::getVar('task') == 'personalizeTheme' || JhaRequest::getVar('task') == 'savePersonalizedChanges')){
       			jhaimport('jhaley.html.theme'); 
	            $renderer = new JhaThemeRenderer();
       		}
       		else {
	            jhaimport('jhaley.html.renderer'); 
	            $renderer = new JhaRenderer();
       		}
       	}
       	elseif($type == 'block'){
       		jhaimport('jhaley.html.block'); 
            $renderer = new JhaRendererBlock();
       	}
       	elseif($type == 'module'){
            .....
        }
        return $renderer;
    }
    
	static function &getDBO(){
        static $dbo;
        if (!is_object($dbo)) {
            jhaimport('jhaley.db.dbo');             
            $dbo = new JhaDBO(JhaFactory::getConfig());
        }
        return $dbo;
    }
    
	static function &getConfig(){
        static $config;
        if (!is_object($config)) {
            require_once JHA_CONFIGURATION_PATH . DS . 'config.php'; 
            $config = new JhaConfiguration();
        }
        return $config;
    }

	.....
}
?>
\end{lstlisting}


Un ejemplo claro del uso del patr\'on Singleton se puede ver en la instancia de la clase \emph{JhaDBO}, la cual es utilizada constantemente en los modelos de los distintos m\'odulos.

\begin{lstlisting}[label=singleton,caption=Patr\'on Singleton en acci\'on]
<?php
defined( '_JHAEXEC' ) or die( 'Access Denied' );
jhaimport('jhaley.mvc.model');

class MenuModelMenu extends JhaModel {
    public function __construct() {
        parent::__construct();
    }
    
    public function getMenus(){
    	$db = &JhaFactory::getDBO();
        $db->setQuery('SELECT m.id, m.titulo, m.descripcion, u.nombre as creador FROM #__menu as m, #__usuario as u WHERE m.usuario = u.id ORDER BY m.id');
        return $db->loadObjectList();
    }
    
	public function getMenu($id){
    	$db = &JhaFactory::getDBO();
        $db->setQuery('SELECT * FROM #__menu WHERE id = ' . $id);
        return $db->loadObject();
    }
    .....
}
?>
\end{lstlisting}


\subsection{Manejo de Archivos compresos}

\begin{figure}[h]
\centering
\includegraphics[scale=.4, keepaspectratio=true]{imagenes/05_imagen.png}
\caption{Modelo de clases: Manejo de Archivos Compresos. [Elaboraci\'on propia].}
\end{figure}

El manejo de archivos compresos esta dado por un grupo de clases especializadas para este cometido, adem\'as estas clases utilizan elementos propios del core de PHP a trav\'es de la manipulaci\'on de los archivos compresos en forma de datos binarios.\\

La superclase \textit{Archive} contiene ciertos elementos que merecen ser detallados.\\

Una de las caractiristicas primordiales es que sus configuraciones est\'an dadas por una serie de opciones:
\begin{itemize}
	\item[\textbf{basedir}], describe el directorio donde se comprimir\'an/descomprimir\'an los archivos.
	\item[\textbf{name}], describe el nombre del archivo compreso.
	\item[\textbf{inmemory}], describe se se manipulara el archivo directamente desde el disco duro o en memoria.
	\item[\textbf{level}], describe el nivel de compresi\'on.
	\item[...] entre otros.
\end{itemize}

\begin{lstlisting}[label=compress_option,caption=Opciones para la configuraci\'on de los compresos.]
<?php
defined( '_JHAEXEC' ) or die( 'Access Denied' );

class Archive {
	function __construct($name) {
		$this->options = array (
			'basedir' => ".",
			'name' => $name,
			'prepend' => "",
			'inmemory' => 0,
			'overwrite' => 0,
			'recurse' => 1,
			'storepaths' => 1,
			'followlinks' => 0,
			'level' => 3,
			'method' => 1,
			'sfx' => "",
			'type' => "",
			'comment' => ""
		);
		$this->files = array ();
		$this->exclude = array ();
		$this->storeonly = array ();
		$this->error = array ();
	}
    .....
}
?>
\end{lstlisting}


Otra caracter\'istica es la posibilidad de crear archivos descargables desde los navegadores.

\begin{lstlisting}[label=compress_downloadable,caption=Opciones para la descarga de los compresos.]
<?php
defined( '_JHAEXEC' ) or die( 'Access Denied' );

class Archive {
    .....
    function downloadFile() {
		if ($this->options['inmemory'] == 0) {
			$this->error[] = "Can only use download_file() if archive is in memory. Redirect to file otherwise, it is faster.";
			return;
		}
		switch ($this->options['type']) {
			case "zip":
				header("Content-Type: application/zip");
			break;
			case "bzip":
				header("Content-Type: application/x-bzip2");
			break;
			case "gzip":
				header("Content-Type: application/x-gzip");
			break;
			case "tar":
				header("Content-Type: application/x-tar");
		}
		$header = "Content-Disposition: attachment; filename=\"";
		$header .= strstr($this->options['name'], "/") ? substr($this->options['name'], strrpos($this->options['name'], "/") + 1) : $this->options['name'];
		$header .= "\"";
		header($header);
		header("Content-Length: " . strlen($this->archive));
		header("Content-Transfer-Encoding: binary");
		header("Cache-Control: no-cache, must-revalidate, max-age=60");
		header("Expires: Sat, 01 Jan 2000 12:00:00 GMT");
		print($this->archive);
	}
}
?>
\end{lstlisting}


%Ejemplo Zip y diferencia con Tar
Las clases \emph{ZipFile} y \emph{TarFile} extienden directamente de \emph{Archive}, mientras que \emph{ZipFile} representa a los archivos compresos completamente formados, \emph{TarFile} representa a un archivo de empaquetado.

\begin{lstlisting}[label=compress_zip,caption=Compresos Zip.]
<?php
defined( '_JHAEXEC' ) or die( 'Access Denied' );
jhaimport('jhaley.compress.archive');

class ZipFile extends Archive {
	function __construct($name) {
		parent::__construct($name);
		$this->options['type'] = "zip";
	}

	function createZip() {
		...
	}
}
?>
\end{lstlisting}


\begin{lstlisting}[label=compress_tar,caption=Empaquetadores Tar.]
class TarFile extends Archive {
	function __construct($name) {
		parent::__construct($name);
		$this->options['type'] = "tar";
	}

	function createTar() {
		...
	}
}
?>
\end{lstlisting}


Los compresos \emph{Gzip}, \emph{TGzip}, \emph{Bzip} y \emph{TBzip} son representados por las subclases \emph{GzipFile} y \emph{BzipFile}. Lo que se gana separando en distintas clases el manejo de los compresos son dos cosas. Primero, que se tienen clases especializadas en un solo tipo de compreso a la vez. Segundo, que es mucho m\'as sencillo a\~nadir nuevos tipos de compresos por medio de la herencia y definici\'on del comportamiento del nuevo compreso.

\subsection{Clases para la Gesti\'on de la Base de Datos}
Las clases que se ven en el siguiente gr\'afico son escenciales para la gesti\'on de la informaci\'on en la base de datos. La clase \textit{JhaDBO} sigue el patr\'on Singleton y es utilizado en todas las llamadas a la base de datos, la clase JhaTable es la superclase para todos los DTOs (Data Transfer Object) que representan a las tablas de la base de datos.\\

\begin{figure}[h]
\centering
\includegraphics[scale=.5, keepaspectratio=true]{imagenes/06_imagen.png}
\caption{Modelo de clases: Gesti\'on de Base de Datos. [Elaboraci\'on propia].}
\end{figure}

A continuaci\'on se describen brevemente las clases dedicadas a la gesti\'on de base de datos.\\

Todo los objetos de \textit{iCMS} tienen la capacidad de redireccionar a otras URLs, esto es bastante \'util sobre todo en la secci\'on de administraci\'on de \textit{i}CMS, tambi\'en cuentan con la capacidad de recuperar sus propiedades internas de los objetos.\\

\begin{lstlisting}[label=jha_object,caption=Objeto Padre.]
<?php
defined( '_JHAEXEC' ) or die( 'Access Denied' );

class JhaObject{
    public function getProperties( $isPublic = true ) {
        $vars  = get_object_vars($this);
        if($isPublic) {
            foreach ($vars as $key => $value) {
                if ('_' == substr($key, 0, 1)) {
                    unset($vars[$key]);
                }
            }
        }
        return $vars;
    }
    
    protected function redirect($url = 'index.php'){
    	$config = JhaFactory::getConfig();
        if (preg_match( '#^index[2]?.php#', $url )) {
            $host = 'http://' . $_SERVER['HTTP_HOST'];
            $site = $config->site;
            $url = $host . $site . $url;
        }
        header( 'HTTP/1.1 301 Moved Permanently' );
        header( 'Location: ' . $url );
    }
}
?>
\end{lstlisting}


\emph{JhaDBO} es un objeto creado con la \'unica finalidad de procesar y ejecutar las consultas SQL nativas, adicionalmente contiene funciones auxiliares para realizar el trabajo de procesamiento de las consultas.

\begin{lstlisting}[label=jha_dbo,caption=DBO (DataBase Object).]
<?php
...

class JhaDBO extends JhaObject {
	//Conexion a la base de datos
	private $_conexion;
	
	//Configuracion del sitio
	private $_config;
	
	//Consulta SQL 
	private $_sql;
	private $_empty;
	
	//Offset y limit, sirven para obtener resultados parciales.
	private $_offset;
	private $_limit;
	private $_cursor;
	
	function __construct($config = null){
		$this->_config = $config;
		if(!$this->open()){
			throw new Exception ( "Error al conectarse a la Base de Datos." );
		}
	}
	...
	
	/**
	 * Abre una conexion a la base de datos, esta conexion se deberia abrir una sola vez.
	 */
	public function open(){
		...
	}
	
  public function close() {
    ...
  }
    
  public function setQuery($query, $offset = 0, $limit = -1){
  	$this->_sql = $this->replacePreffix($query);
  	$this->_offset = $offset;
  	$this->_limit = $limit;
  }
  
  /**
   * Establece la consulta SQL y luego la ejecuta
   */
  public function executeQuery($query){
  	...
  }
  
  private function replacePreffix($query){
  	return str_replace("#__", $this->_config->db_preffix, $query);
  }
  
  /**
   * Ejecuta la consulta SQL
   */
  private function _query(){
  	...
  }
  ...
  
  /**
   * Tras la ejecucion de la consulta SQL, este metodo retorna el resultado como un Objeto
   */
  public function loadObject() {
    ...
  }
  
  /**
   * Tras la ejecucion de la consulta SQL, este metodo retorna el resultado como una lista de Objetos
   */
  public function loadObjectList( $key = '' ) {
    ...
  }
  
  /**
   * Inserta los datos del objeto en la base de datos.
   */
  public function insertObject( $table, &$object, $keyName = NULL ) {
    ...
  }
  
  /**
   * Actualiza la informacion  de la base da datos con la informacion del objeto
   */
  public function updateObject( $table, &$object, $keyName, $updateNulls = true ) {
    ...
  }
  
  /**
   * Elimina el registro descrito por el objeto
   */
  public function deleteObject( $table, &$object, $keyName = NULL ) {
    ...
  }

  public function insertid() {
    return mysql_insert_id($this->_conexion);
  }
  
  /**
   * Inserta registros en la base de datos con mas de un valor como clave primaria.
   */
  public function insertGeneralObject( $table, $elems) {
    ...
  }
  
  /**
   * Elimina registros de la base de datos que tienen mas de un valor como clave primaria.
   */
  public function deleteGeneralObject( $table, $elems) {
    ...
  }
}
?>
\end{lstlisting}


\subsection{Clases para la Gesti\'on de Archivos y Directorios}
Las clases para la gesti\'on de los archivos y directorios se muestran en el siguiente diagrama. La clase \textit{JhaFile} sirve principalmente para la lectura y escritura de archivos, la clase \textit{JhaDirectory} provee una estructura para gestionar los archivos y subdirectorios de un directorio.\\

\begin{figure}[h]
\centering
\includegraphics[scale=.5, keepaspectratio=true]{imagenes/07_imagen.png}
\caption{Modelo de clases: Gesti\'on de Archivos y Directorios. [Elaboraci\'on propia].}
\end{figure}

A continuaci\'on se describen brevemente las clases dedicadas a la gesti\'on de archivos y directorios.\\

La clase \emph{JhaFile} permite interactuar de forma directa con los diferentes archivos a trav\'es de sus m\'etodos \emph{read} y \emph{write}.\\

\begin{lstlisting}[label=jha_file,caption=JhaFile.]
<?php
...

class JhaFile extends JhaObject {
	var $_pathfile;
	var $_content;
	
	public function __construct($pathfile = '', $content = ''){
		...
	}
	
	public function setPathFile($pathfile){
		$this->_pathfile = $pathfile;
	}
	
	public function setContent($content){
		$this->_content = $content;
	}
	
	public function getContent(){
		return $this->_content;
	}
	
	/**
	 * Se encarga de escribir el archivo.
	 */
	public function write(){
		...
	}
	
	/**
	 * Se encarga de leer el archivo.
	 */
	public function read(){
		...
	}
}
?>
\end{lstlisting}


La clase \emph{JhaDirectory} se combina con los objetos de tipo \emph{JhaFile} para permitir interactuar tanto con los diferentes archivos como con los directorios a trav\'es de sus m\'etodos \emph{getFolders} y \emph{getFiles}, ademas cuenta con una funci\'on extra para validar las extensiones de los archivos \emph{validateExtension}.\\

\begin{lstlisting}[label=jha_directory,caption=JhaDirectory.]
<?php
...

class JhaDirectory extends JhaObject {
	var $_pathfile;
	var $_files;
	
	public function __construct($pathfile = '.'){
		...
	}
	
	public function setPathFile($pathfile){
		$this->_pathfile = $pathfile;
	}
	
	public function read(){
        $this->_files = scandir($this->_pathfile);
	}
	
	/**
	 * En funcion al filepath recibido, retorna todos los directorios existentes en el.
	 */
	public function getFolders(){
		...
	}
	
	/**
	 * En funcion al filepath recibido, retorna todos los archivos existentes en el.
	 * No toma en cuenta los archivos de los directorios.
	 */
	public function getFiles($filetypes = array()){
		...
	}
	
	/**
	 * Encargado de validar que todos los archivos del directorio tengan la misma extension
	 */
	private function validateExtension($file, $filetypes){
		...
	}
}
?>
\end{lstlisting}


\subsection{Clases para la Gesti\'on de Elementos Web}
Las clases para la gesti\'on de los elementos web se muestran en el siguiente diagrama. Estas clases estan basadas en el Framework Goaamb (desarrollado por el Ingeniero de Sistemas Alvaro Michel Barrera).\\
Todas estas clases han sido pensadas principalmente para trabajar con generadores de c\'odigo, pero en el proyecto iCMS se hace uso de ellas principalmente en el manejo de archivos XML y JS.\\

\begin{figure}[h]
\centering
\includegraphics[scale=.4, keepaspectratio=true]{imagenes/08_imagen.png}
\caption{Modelo de clases: Gesti\'on de Elementos Web. [Elaboraci\'on propia].}
\end{figure}

Como podemos ver en la figura anterior, la super clase \emph{Tag} contienen la mayor parte de las funciones propias para los elementos HTML y PHP. Asimismo muchos de los elementos HTML est\'an representados por las clases: \emph{HtmlTag}, \emph{InputTag}, \emph{Select}, \emph{Option} y \emph{Tabla}. Por otro lado tenemos a la clase para el manejo de documentos XML: \emph{XmlTag}. Para el manejo de c\'odigos PHP, tenemos la clase \emph{PhpTag}; dado que los elementos JavaScript tienen cierto parecido con los tags php, la clase JSTag extiende de esta \'ultima.\\
Los elementos JSON se comportan de una manera muy distinta a los elementos que tienen ``tags'', raz\'on por lo cual la clase \emph{JSON} no extiende de ninguna otra.

\subsection{Clases de la Arquitectura del \textit{i}CMS}
Las clases que se muestran a continuaci\'on son las superclases que representan al patr\'on Modelo-Vista-Controlador. Las implementaciones de estas superclases estan presentes en las distintas extensiones (m\'odulos).\\

\begin{figure}[h]
\centering
\includegraphics[scale=.65, keepaspectratio=true]{imagenes/09_imagen.png}
\caption{Modelo de clases: Modelo-Vista-Controlador. [Elaboraci\'on propia].}
\end{figure}

La clase \emph{JhaModel} describe dos m\'etodos importantes como \emph{getTable} y \emph{getInstance}; \emph{getTable} se encarga de recuperar objetos de tipo \emph{JhaTable} y es principalmente utilizada a la hora de recuperar registros de la base de datos. Por otro lado \emph{getInstance} retorna una instancia del modelo en cuesti\'on, es decir que depende del m\'odulo desde el que se llame a este m\'etodo.

\begin{lstlisting}[label=jha_model,caption=JhaModel.]
<?php
...

class JhaModel extends JhaObject {
	
	protected $_name;
	protected $_module;
	
    public function __construct(){
        if (empty( $this->_name ) || empty( $this->_module )) {
            $this->_name = $this->getName();
            $this->_module = $this->getModule();
        }
    }
    
    public function &getTable($name = '') {
    	...
    }
    ...
    
    public function &getInstance($name, $prefix){
        ...
    }
}
?>
\end{lstlisting}


La clase \emph{JhaView} se encarga de renderizar las diferentes plantillas dependiendo del controlador y la acci\'on relacionada a estos, en resumen por cada acci\'on que ejecute el controlador, la vista puede renderizar diferentes pantallas, es por eso que cada vista tiene 'n' plantillas, a continuaci\'on se describen los m\'etodos m\'as importantes:
\begin{description}
\item[display] Se encarga de mostrar plantilla por defecto para la vista.
\item[getModel] Retorna el modelo para el m\'odulo al que pertenece la vista.
\item[setLayout] Esteblece la plantilla que se utilizara para renderizar el resultado de ejecutar la acci\'on en el controlador.
\item[assignRef] Esteblece referencias a los valores recuperados desde el modelo, en s\'intesis a\~nade nuevos atributos a la vista, de forma que la plantilla los pueda utilizar de forma casi transparente.
\item[getInstance] Devuelve una instancia de la vista, al igual que con el modelo, esta acci\'on depende del contexto.
\end{description}

\begin{lstlisting}[label=jha_view,caption=JhaView.]
<?php
...

class JhaView extends JhaObject {

    protected $_name;
    protected $_module;
    protected $_layout;
    protected $_ext;
    protected $_model;
    
    public function __construct(){
        if (empty( $this->_name ) || empty( $this->_module )) {
            $this->_name = $this->getName();
            $this->_module = $this->getModule();
        }
        $this->_layout = $this->_name;
        $this->_ext = '.php';
    }
    
    public function display(){
    	$file = $this->_layout . $this->_ext;
   	    $content = '';
   	    $urlFile = $GLOBALS['JHA_MODULE_PATH'].'views'.DS.'html'.DS.$file;
   	    if(file_exists( $urlFile )){
   	    	ob_start();
            require_once $urlFile;
            $content = ob_get_contents();
            ob_end_clean();
            echo $content;
        }
        else{
        	throw new Exception("Vista no encontrada");
        }
    }
    
    public function &getModel($model = ''){
    	$model = (empty($model) ? $this->getName() : $model);
        $prefix = $this->_module . 'Model';
        jhaimport('jhaley.mvc.model');
        $this->model = JhaModel::getInstance($model, $prefix);
        return $this->model;
    }
    ...
    
    function assignRef($key, &$val) {
        if (is_string($key) && substr($key, 0, 1) != '_') {
            $this->$key =& $val;
            return true;
        }
        return false;
    }
    
    public function &getInstance($name, $prefix){
        $name = preg_replace('/[^A-Z0-9_\.-]/i', '', $name);
        $prefix = preg_replace('/[^A-Z0-9_\.-]/i', '', $prefix);
		    $viewClass	= $prefix.$name;
		    $result = false;
		    if (!class_exists( $viewClass )) {
			    $path = $GLOBALS['JHA_MODULE_PATH'].'views'.DS.$name.'.php';
			    if ($path) {
				    require_once $path;
				    if (!class_exists( $viewClass )) {
					    throw new Exception( 'Vista ' . $viewClass . ' no encontrado.' );
					    return $result;
				    }
			    }
			    else return $result;
		    }
		    $result = new $viewClass();
		    return $result;
    }
}
?>
\end{lstlisting}


La clase \emph{JhaController} se encarga de ejecutar las acciones tr\'as las cuales se delega la responsabilidad de renderizar el resultado a alguna de las vistas, a continuaci\'on se describen los m\'etodos m\'as importantes de esta clase:
\begin{description}
\item[display] Se encarga de hacer que la vista muestre su plantilla por defecto.
\item[execute] Ejecuta el m\'etodo correspondiente a la acci\'on a ejecutarse, por defecto ejecuta la acci\'on ``display''.
\item[getView] Retorna la vista asociada al controlador.
\item[getModel] Retorna el modelo para el m\'odulo al que pertenece la vista.
\end{description}

\begin{lstlisting}[label=jha_controller,caption=JhaController.]
<?php 
...

class JhaController extends JhaObject {
	
	  var $_tasks;
	  var $_name;
	  var $_module;
	
    public function __construct(){
    	  ...
    }
    
    public function display(){
      	$view = &$this->getView();
      	$model = &$this->getModel();
      	$view->setModel($model);
      	$view->display();
    }
    
    public function execute($task){
        $task = strtolower( $task );
        if (isset($this->_tasks[$task])) {
        	$jobToDo = $this->_tasks[$task];
        }
        elseif (isset($this->_tasks['display'])){
        	$jobToDo = $this->_tasks['display'];
        }
        else{
        	throw new Exception( 'Proceso asociado a la tarea ' . $task . ' no encontrada.' );
        }
        return $this->$jobToDo();
    }
    
    public function &getView($view = ''){
    	$view = (empty($view) ? $this->getName() : $view);
        $prefix = $this->_module . 'View';
        jhaimport('jhaley.mvc.view');
        return JhaView::getInstance($view, $prefix);
    }
    
    public function &getModel($model = ''){
    	$model = (empty($model) ? $this->getName() : $model);
        $prefix = $this->_module . 'Model';
        jhaimport('jhaley.mvc.model');
        return JhaModel::getInstance($model, $prefix);
    }
    ...
}
?>
\end{lstlisting}


\newpage
\subsection{Clases para la Gesti\'on de Renderizadores}
Las clases que se muestran en el siguiente diagrama, se utilizan para renderizar alg\'un tipo de contenido en particular, se tienen rederizadores para los bloques, los m\'odulos, el men\'u de administraci\'on, las plantillas, etc.\\
Si bien todos los renderizadores tienen el m\'etodo ``render'' cada renderizador tiene un comportamiento totalmente diferente al de otro.

\begin{figure}[h]
\centering
\includegraphics[scale=.45, keepaspectratio=true]{imagenes/10_imagen.png}
\caption{Modelo de clases: Renderizadores. [Elaboraci\'on propia].}
\end{figure}

A continuaci\'on vemos en detalle cada uno de los renderizadores.\\

El renderizador \textit{JhaRenderer} hereda de la clase \textit{JhaObject}, este se encarga de renderizar la plantilla entera, es decir que de alguna manera este hace uso de los dem\'as renderizadores.\\
Los siguientes m\'etodos son parte de este renderizador:
\begin{description}
\item[render] B\'asicamente este m\'etodo es el punto de partida para la renderizaci\'on.
\item[loadTemplate] Retorna la plantilla como si fuera una cadena para hacerla m\'as manejablea nivel de c\'odigo.
\item[renderTemplate] Recibe como parametro el resultado de ``loadTemplate'', en base a este parametro agrega contenidos que son los resultados de los otros renderizadores.
\item[getBuffer] Este m\'etodo act\'ua como discriminador, dependiendo del tipo de contenido que debe mostrarse llama a uno u otro renderizador. En el caso de los bloques hace una comprobaci\'on adicional que le permite saber si un bloque se debe mostrar o no a trav\'es de los m\'etodos ``canShowed'' e ``isRenderable''.
\item[getBlocks] Es un m\'etodo auxiliar para ``getBuffer''.
\end{description}

\begin{lstlisting}[label=jha_renderer,caption=Renderizador JhaRenderer.]
<?php
...

class JhaRenderer extends JhaObject {
	public function render(){
		$db = &JhaFactory::getDBO();
        $db->setQuery('SELECT * FROM #__plantilla WHERE predeterminado = 1');
        $row = $db->loadObject();
        
		$template = $this->loadTemplate($row);
		echo $this->renderTemplate(str_replace('<jhadoc:include type="maincontent" />', $row->html, $template));
	}
	 
	protected function loadTemplate($row){
        $template = '';
		ob_start();
        require_once JHA_THEMES_PATH.DS.($row ? $row->nombre : 'default').DS.'index.php';
        $template = ob_get_contents();
        ob_end_clean();
        return $template;
	}
	
	protected function getBuffer($type = null, $region = null){
		$content = '';
		if($type == null){
		    return;
		}
		$renderer = &JhaFactory::getRenderer($type);
		if($type == 'module'){
			$GLOBALS['JHA_MODULE_PATH'] = JHA_BASE_PATH.DS.'modules'.DS.JhaRequest::getVar('elem','mod_content').DS;
			$path = $GLOBALS['JHA_MODULE_PATH'].substr(JhaRequest::getVar('elem', 'mod_content'),4).'.php';
			$content = $renderer->render($path).$content;
		}
		elseif ($type == 'block' && $region != null){
			$blocks = $this->getBlocks($region);
			if($blocks) {
				if(JhaUtility::userCanEdit()){
				    $content .= '<div class="jhablock"><div class="jhablock-header">Region: ' . ucfirst($region) . '</div>';
				    $script = '';
		        }
				foreach($blocks as $block) {
					if($this->isRenderable($block) && $this->canShowed($block)){
						$GLOBALS['JHA_BLOCK_PATH'] = JHA_BASE_PATH.DS.'blocks'.DS.$block->renderizador.DS;
						$path = $GLOBALS['JHA_BLOCK_PATH'] . substr($block->renderizador,6) . '.php';
						$content .= $renderer->render($path, $block);
						if(JhaUtility::userCanEdit()){
							$script .= "dragBlock.addTarget(Jha.dom.$('block" . $block->id . "'));\n";
						}
					}
				}
				if(JhaUtility::userCanEdit()){
					//aumentar final de bloque y tb. la opcion de Add more blocks
					//la opcion se abrira en un iframe.
					$content .= JhaHTML::script($script, false);
					//agregar el enlace para agregar nuevos bloques.
					$content .= (JhaUtility::userCanAdministrate() ? '<div><a href="index.php?elem=mod_base&controller=block&task=newblock&region=' . $region . '"><img src="images/new.jpg" title="Agregar nuevo bloque" /></a></div>' : '');
					$content .= '</div>';
				}
			}
			else {
				if(JhaUtility::userCanEdit()){
					$content .= '<div class="jhablock"><div class="jhablock-header">Region: ' . ucfirst($region) . '</div>' . (JhaUtility::userCanAdministrate() ? '<div><a href="index.php?elem=mod_base&controller=block&task=newblock&region=' . $region . '"><img src="images/new.jpg" title="Agregar nuevo bloque" /></a></div>' : '') . '</div>';
				}
			}
		}
		elseif ($type == 'head' || ($type == 'admin-menu' && JhaUtility::userCanAdministrate())){
            $content = $renderer->render().$content;
        }
		return $content;
	}
	
	protected function getBlocks($region){
		$db = &JhaFactory::getDBO();
        $db->setQuery("SELECT * FROM #__bloque WHERE region = '" . $region . "' ORDER BY orden ASC");
        return $db->loadObjectList();
	}
	
	protected function canShowed($block){
		if(JhaUtility::userCanEdit()){
			return true;
		}
		return 0 == intval($block->needlogin);
	}
	
	protected function isRenderable($block){
		$db = &JhaFactory::getDBO();
        $itemid = JhaRequest::getVar('itemid',0);
        if($itemid == 0){
            $db->setQuery('SELECT * FROM #__menuitem WHERE home = 1');
            $row = $db->loadObject();
            $itemid = $row->id;
        }
        $db->setQuery("SELECT * FROM #__menubloque WHERE (idmenu = '" . $itemid . "' OR idmenu = '0' ) AND idbloque = '" . $block->id . "'");
        $row = $db->loadObject();
        if($row)
            return true;
        return false;
	}
	
	protected function renderTemplate($template) {
        if(JhaUtility::userCanEdit()){
        	$GLOBALS['JHA_HEAD_VARS'][] = JhaHTML::script("dragBlock = new Jha.drag();\ndragBlock.setType('block');\ndragBlock.ajaxPost = function (objajax, idSource, idTarget, isBeforeTarget) { objajax.json = true; res = { elem : 'mod_base', controller : 'block', src : idSource, trg : idTarget, json : objajax.json, before : isBeforeTarget, task : 'reorder'}; return res; };", false);
        }
		$replace = array();
        $matches = array();
        if(preg_match_all('#<jhadoc:include\ type="([^"]+)" (.*)\/>#iU', $template, $matches)) {
            $matches[0] = array_reverse($matches[0]);
            $matches[1] = array_reverse($matches[1]);
            $matches[2] = array_reverse($matches[2]);
            $count = count($matches[1]);
            for($i = 0; $i < $count; $i++) {
                $attribs = JhaUtility::parseAttributes( $matches[2][$i] );
                $replace[$i] = $this->getBuffer($matches[1][$i], (isset($attribs['region']) ? $attribs['region'] : null));
            }
            $template = str_replace($matches[0], $replace, $template);
        }
        return $template;
	} 
}
?>
\end{lstlisting}


El renderizador \textit{JhaRendererModule} tambi\'en hereda de la clase \textit{JhaObject}, este se encarga de renderizar los m\'odulos unicamente.\\
Este renderizador solamente cuenta con el m\'etodo render ya que no necesita hacer ning'un tipo de comprobaci\'on, de esto se encargan los propios m\'odulos.

\begin{lstlisting}[label=jha_renderer_module,caption=Renderizador de modulos.]
<?php
...

class JhaRendererModule extends JhaObject {
    public function render($path){
        $content = '';
        if(file_exists( $path )){
            ob_start();
            require $path;
            $content = ob_get_contents();
            ob_end_clean();
        }
        return $content;
    }
}
?>
\end{lstlisting}


El renderizador \textit{JhaRendererBlock} al igual que los anteriores, tambi\'en hereda de la clase \textit{JhaObject}, este se encarga de renderizar unicamente los bloques, algo que cabe destacar es que cada bloque tienen un conjunto de configuraciones para lo cual cuenta con un m\'etdo auxiliar para este menester.\\
A continuaci\'on vemos una descripci\'on de los m\'etodos que son parte de este renderizador:
\begin{description}
\item[render] En escencia se encarga de devolver el resultado de renderizar alg\'un bloque.
\item[getParams] Retorna los parametros para el bloque, esta informaci\'on se la obtiene de la base de datos juntamente con el contenido del bloque.
\end{description}

\begin{lstlisting}[label=jha_renderer_block,caption=Renderizador de bloques.]
<?php
...

class JhaRendererBlock extends JhaObject {
	public function render($path, $block){
		$content = '';
		$params = $this->getParams($block);
    if(file_exists( $path )){
      ob_start();
      require $path;
      if(JhaUtility::userCanEdit()){
        $content .= '<div class="block blockelement' . $params['suffix'] . '" id="block' . $block->id . '"><div class="block-header"><div align="left" style="float:left;"><a href="index.php?elem=mod_base&controller=block&task=editblock&id=' . $block->id . '" title="Editar"><img src="images/edit.jpg" /></a></div><div align="left" onmousedown="javascript:dragBlock.onDragStart(event, Jha.dom.$(\'block' . $block->id . '\'));" style="line-height: 2;">' . $block->titulo . '</div></div>' . ob_get_contents() . '</div>';
      }
      else {
        $content = '<div class="blockelement' . $params['suffix'] . '">' . ob_get_contents() . '</div>';
      }
      ob_end_clean();
    }
    return $content;
	}
	
	public function getParams($block){
		$params = split("\n",$block->params);
		$res = array();
		foreach ($params as $param){
			$parts = split('=', $param);
			$res[$parts[0]] = $parts[1];
		}
		return $res;
	}
}
?>
\end{lstlisting}


El renderizador \textit{JhaRendererHead} se encarga de renderizar unicamente elementos que van en el tag HTML ``head''.\\
A continuaci\'on vemos una descripci\'on de los m\'etodos que son parte de este renderizador:
\begin{description}
\item[render] Renderiza los las hojas de estilo (CSS) y los scripts (JS) dependiendo del tipo de sesi\'on.
\item[cleanHeadVars] Esto limpia los elementos duplicados, adem\'as crea un orden entre los elementos que son parte del tag ``tag''.
\end{description}

\begin{lstlisting}[label=jha_renderer_head,caption=Renderizador para el tag HTML `head'.]
<?php
...

class JhaRendererHead extends JhaObject {
    public function render(){
        $content = '';
        $GLOBALS['JHA_HEAD_VARS'] = (isset($GLOBALS['JHA_HEAD_VARS'][0]) ? $GLOBALS['JHA_HEAD_VARS'] : array());
        if(JhaUtility::userCanEdit()){
        	$GLOBALS['JHA_HEAD_VARS'] = array_reverse($GLOBALS['JHA_HEAD_VARS']);
        	$GLOBALS['JHA_HEAD_VARS'][] = JhaHTML::script('libraries/jhaley/js/menu.js');
        	$GLOBALS['JHA_HEAD_VARS'][] = JhaHTML::script('libraries/jhaley/js/popup.js');
        	$GLOBALS['JHA_HEAD_VARS'][] = JhaHTML::script('libraries/jhaley/js/pmenu.js');
        	$GLOBALS['JHA_HEAD_VARS'][] = JhaHTML::script('libraries/jhaley/js/effect.js');
        	$GLOBALS['JHA_HEAD_VARS'][] = JhaHTML::script('libraries/jhaley/js/drag.js');
        	$GLOBALS['JHA_HEAD_VARS'][] = JhaHTML::script('libraries/jhaley/js/ajax.js');
	        $GLOBALS['JHA_HEAD_VARS'][] = JhaHTML::script('libraries/jhaley/js/jha.js');
	        $GLOBALS['JHA_HEAD_VARS'][] = JhaHTML::stylesheet('libraries/jhaley/css/menu.css');
	        $GLOBALS['JHA_HEAD_VARS'][] = JhaHTML::stylesheet('libraries/jhaley/css/base.css');
	        $GLOBALS['JHA_HEAD_VARS'] = array_reverse($GLOBALS['JHA_HEAD_VARS']);
        }
        else{
        	$GLOBALS['JHA_HEAD_VARS'] = array_reverse($GLOBALS['JHA_HEAD_VARS']);
        	$GLOBALS['JHA_HEAD_VARS'][] = JhaHTML::script('libraries/jhaley/js/effect.js');
	        $GLOBALS['JHA_HEAD_VARS'][] = JhaHTML::script('libraries/jhaley/js/jha.js');
	        $GLOBALS['JHA_HEAD_VARS'] = array_reverse($GLOBALS['JHA_HEAD_VARS']);
        }
        if(isset($GLOBALS['JHA_HEAD_VARS']) && count($GLOBALS['JHA_HEAD_VARS']) > 0){
        	$this->cleanHeadVars();
            $content .= implode('',$GLOBALS['JHA_HEAD_VARS']);
        }
        return $content;
    }
    
    protected function cleanHeadVars(){
    	$array = array();
    	foreach ($GLOBALS['JHA_HEAD_VARS'] as $headVar){
    		if (JhaUtility::inArray($headVar, $array) == -1) {
    			$array[] = $headVar;
    		}
    	}
    	$GLOBALS['JHA_HEAD_VARS'] = $array;
    }
}
?>
\end{lstlisting}


El renderizador \textit{JhaAJAXRenderer} se encarga de renderizar unicamente bloques y/o m\'odulos. Este renderizador no sigue el flujo normal que es pasar por la clase \textit{JhaRenderer}, sino que como su nombre indica esta dise\~nado para las peticiones as\'incronas.\\
Los m\'etodos que son parte de este renderizador son:
\begin{description}
\item[render] Renderiza bloques y m\'odulos dependiendo de la petici\'on AJAX.
\item[getBlock] Recupera los datos del bloque, si es que la petici\'on AJAX est\'a relacionada con los bloques.
\end{description}

\begin{lstlisting}[label=jha_renderer_ajax,caption=Renderizador las llamadas as\'incronas.]
<?php
...

class JhaAJAXRenderer extends JhaObject {
    public function render(){
        $content = '';
        $elem = JhaRequest::getVar('elem');
        if(substr($elem, 0, 3) == 'mod'){
        	$renderer = &JhaFactory::getRenderer('module');
        	$GLOBALS['JHA_MODULE_PATH'] = JHA_BASE_PATH.DS.'modules'.DS.$elem.DS;
            $path = $GLOBALS['JHA_MODULE_PATH'].substr($elem, 4).'.php';
            $content = $renderer->render($path).$content;
        }
        else {
        	$block = $this->getBlock();
        	$renderer = &JhaFactory::getRenderer('block');
        	$GLOBALS['JHA_BLOCK_PATH'] = JHA_BASE_PATH.DS.'blocks'.DS.$block->renderizador.DS;
            $path = $GLOBALS['JHA_BLOCK_PATH'] . substr($block->renderizador,6) . '.php';
            $content = $renderer->render($path, $block);
        }
        echo $content;
    }
    
    protected function getBlock(){
    	$db = &JhaFactory::getDBO();
        $db->setQuery("SELECT * FROM #__bloque WHERE id = '" . JhaRequest::getVar('id') . "'");
        return $db->loadObject();
    }
}
?>
\end{lstlisting}


El renderizador \textit{JhaRendererAdminMenu} al igual que los anteriores, tambi\'en hereda de la clase \textit{JhaObject}, este renderizador est\'a destinado unicamente a renderizar el menu de administraci\'on.\\
Los m\'etodos que son parte de este renderizador son:
\begin{description}
\item[render] Renderiza los menus y sub menus de administraci\'on, el resto de m\'etodos son auxiliares para renderizar fragmentos del menu: \textit{createMenu}, \textit{createSubMenuBD}, \textit{getModuleList} y \textit{getAttributes}.
\end{description}

\begin{lstlisting}[label=jha_renderer_head,caption=Renderizador para el tag HTML `head'.]
<?php
...

class JhaRendererAdminMenu extends JhaObject {
  public function render(){
  	$content = '';
    $user = (isset($_SESSION['USER']) ? $_SESSION['USER'] : NULL);
    $canEdit = $user->rol == 'Super Administrador' || $user->rol == 'Editor';
    if($canEdit){
    	$xml = simplexml_load_file(JHA_LIBRARIES_PATH.DS.'jhaley'.DS.'xml'.DS.'adminmenu.xml');
    	$content .= '<div id="jha-admin-menu"><div style="width: 950px; margin: 0pt auto; height: 25px; background-color: black;">' . $this->createMenu($xml) . '</div></div><br />';
    	$GLOBALS['JHA_HEAD_VARS'][] = JhaHTML::script('document.menu = null;
      	window.onload = function(){
    	    element = Jha.dom.$(\'jhamenu\');
    	    var menu = new Jha.menu(element);
    	    document.menu = menu;
        };', false);
    }
    return $content;
  }
  
  protected function createMenu($xml){
  	$content = '<ul' . $this->getAttributes($xml->attributes()) . '>';
 		foreach ($xml->elements->element as $element) {
 			$att = $element->attributes();
 			$content .= '<li><a' . $this->getAttributes($att) . '>' . $element->text . '</a>';
 			if(!isset($element->elements) && $att['type'] == 'bd'){
		    $content .= $this->createSubMenuBD(0);
 			}
 			elseif(isset($element->elements)){
 				$content .= $this->createMenu($element);
 			}
 			$content .= '</li>';
 		}
  	return $content . '</ul>';
  }
  
  protected function createSubMenuBD($level, $id = NULL){
  	$content = '';
  	$links = $this->getModuleList($level, $id);
  	if(count($links) > 0){
  		$content .= '<ul>';
    	foreach ($links as $link) {
    		$href = ($link->link != '' ? ' href="' . $link->link . '&itemid=' . $_SESSION["itemid"] . '"' : '');
    		$content .= '<li><a' . $href . '>' . $link->nombre . '</a>';
    		$content .= $this->createSubMenuBD($level + 1, $link->id) . '</li>';
    	}
    	$content .= '</ul>';
  	}
  	return $content;
  }
  
  protected function getModuleList($level, $id = NULL){
  	$db = &JhaFactory::getDBO();
  	$db->setQuery('SELECT * FROM #__modules WHERE parent = ' . ($level == 0 ? '0' : $id));
    return $db->loadObjectList();
  }
  
  protected function getAttributes($attributes){
  	$attr = '';
  	if(count($attributes) > 0){
  		foreach ($attributes as $index => $value){
  			if($index == 'href'){
  				$attr .= ' ' . $index . '="' . $value . '&itemid=' . $_SESSION["itemid"] . '"';
  			}
  			elseif($index != 'type'){
			    $attr .= ' ' . $index . '="' . $value . '"';
  			}
  		}
  	}
  	return $attr;
  }
}
?>
\end{lstlisting}


El renderizador \textit{JhaThemeRenderer} tambi\'en hereda de la clase \textit{JhaObject}, este renderizador est\'a destinado unicamente a renderizar la plantilla en modo de personalizaci\'on, esto significa que podremos personalizarla desde la interfaz gr\'afica.\\
Los m\'etodos que son parte de este renderizador son:
\begin{description}
\item[render] Renderiza la plantilla de forma muy similar a \textit{JhaRenderer} con la diferencia de que no renderiza el contenido sino que inserta en su lugar elementos para su manipulaci\'on desde la interfaz gr\'afica, es decir para personalizar la plantilla.
\end{description}

\begin{lstlisting}[label=jha_renderer_head,caption=Renderizador para el tag HTML `head'.]
<?php
...

class JhaThemeRenderer extends JhaObject {
	public function render(){
		$db = &JhaFactory::getDBO();
        $db->setQuery('SELECT * FROM #__plantilla WHERE predeterminado = 1');
        $row = $db->loadObject();
        
		$template = $this->loadTemplate($row);
		$_SESSION['themeHTML'] = isset($_SESSION['themeHTML']) ? $_SESSION['themeHTML'] : $row->html;
		$_SESSION['themeXML'] = isset($_SESSION['themeXML']) ? $_SESSION['themeXML'] : $row->xml;
		echo $this->renderTemplate($template);
	}
	 
	protected function loadTemplate($row){
        $template = '';
		ob_start();
        require_once JHA_THEMES_PATH.DS.($row ? $row->nombre : 'default').DS.'index.php';
        $template = ob_get_contents();
        ob_end_clean();
        return $template;
	}
	
	protected function getBuffer($type = null, $region = null){
		$content = '';
		if($type == null){
		    return;
		}
		$renderer = &JhaFactory::getRenderer($type == 'maincontent' ? 'module' : $type);
		if($type == 'maincontent') {
			$GLOBALS['JHA_MODULE_PATH'] = JHA_BASE_PATH.DS.'modules'.DS.JhaRequest::getVar('elem','mod_content').DS;
			$path = $GLOBALS['JHA_MODULE_PATH'].substr(JhaRequest::getVar('elem', 'mod_content'),4).'.php';
			$content = $renderer->render($path).$content;
		}
		elseif ($type == 'head'){
            $content = $renderer->render().$content;
        }
        elseif ($type == 'admin-menu'){
        	$content = $this->renderMenu().$content;
        }
		return $content;
	}
	
	private function renderMenu(){
		$content = '';
		if(JhaUtility::userCanEdit()) {
        	$content .= '<div id="jha-admin-menu"><div style="width: 950px; margin: 0pt auto; height: 25px; background-color: black;"><ul id="jhamenu"><li><a href="index.php?elem=mod_base&controller=theme&task=savePersonalizedChanges">Guardar Cambios</a></li><li><a href="index.php?elem=mod_base&controller=theme&task=cancelPersonalizedChanges">Cancelar</a></li></ul></div></div><br />';
        	$GLOBALS['JHA_HEAD_VARS'][] = JhaHTML::script('document.menu = null;
        	window.onload = function(){
        	    element = Jha.dom.$(\'jhamenu\');
        	    var menu = new Jha.menu(element);
        	    document.menu = menu;
   	        };', false);
        }
        return $content;
	}
	
	protected function renderTemplate($template) {
		$replace = array();
        $matches = array();
        if(preg_match_all('#<jhadoc:include\ type="([^"]+)" (.*)\/>#iU', $template, $matches)) {
            $matches[0] = array_reverse($matches[0]);
            $matches[1] = array_reverse($matches[1]);
            $count = count($matches[1]);
            for($i = 0; $i < $count; $i++) {
                $replace[$i] = $this->getBuffer($matches[1][$i]);
            }
            $template = str_replace($matches[0], $replace, $template);
        }
        return $template;
	} 
}
?>
\end{lstlisting}


La clase \textit{JhaHTML} no es precisamente un renderizador de contenido especializado como el de los m\'odulos o bloques, pero en su lugar es capaz de generar contenido HTML simple como tags `link', `script' y `title'.\\
Los m\'etodos que son parte de este renderizador son:
\begin{description}
\item[stylesheet] Renderiza un tag `link' con alguna hoja de estilos.
\item[script] Renderiza un tag `script' con alg\'un contenido javascript ya sea contenido en linea o desde un archivo.
\item[title] Renderiza un tag `title' con el titulo de la p\'agina.
\item[tagHTML] Renderiza un tag HTML arbitrario.
\item[renderControls] Renderiza el conjunto de controles de administraci\'on de contenidos, es decir los controles de Nuevo, Editar, Eliminar, Guardar, Cancelar. En el panel de administraci\'on estos controles son muy comunes, as\'i que este m\'etodo facilita mucho su creaci\'on
\end{description}

\begin{lstlisting}[label=jha_renderer_head,caption=Renderizador para el tag HTML `head'.]
<?php
...

class JhaHTML extends JhaObject {
    
  public static function stylesheet($url){
  	jhaimport('jhaley.web.tags');
    $taghtml = new Tag ('link');
    $taghtml->setAttribute('type', 'text/css');
    $taghtml->setAttribute('rel', 'stylesheet');
    $taghtml->setAttribute('href', $url);
    return $taghtml->html ();
  }
    
  public static function script($src, $isUrl = true){
  	jhaimport('jhaley.web.jstag');
    $tagjs = new JSTag ();
    if($isUrl){
      $tagjs->setAttribute('type', 'text/javascript');
      $tagjs->setAttribute('src', $src);
    }
    else{
      $tagjs->add($src);
    }
    return $tagjs->html ();
  }
  
  public function title($title = ''){
  	jhaimport('jhaley.web.tags');
  	$taghtml = new Tag ('title', $title);
  	return $taghtml->html ();
  }
  
  //array-> titulos, tasks, linktype**, icons.  
  // ** -> Para la validacion de los checkboxes.
  public function renderControls($titles, $tasks, $linktypes, $icons){
  	if(count($titles) <= 0 && count($tasks) <= 0 && count($linktypes) <= 0) return '';
  	jhaimport('jhaley.web.tags');
  	$tagtable = new Tag ('table');
  	$tagtable -> add( $tagtr = new Tag('tr') );
    for($i = 0; $i < count($titles); $i++){
    	$mustValidate = $linktypes[$i] == 'edit' || $linktypes[$i] == 'delete' || $linktypes[$i] == 'save';
    	$onclick = '';
    	if($mustValidate) {
    		$condicion = ($linktypes[$i] == 'edit' || $linktypes[$i] == 'delete' ? 'Jha.html.checkbox.validate()' : 'validateForm()');
    		$onclick = 'if(' . $condicion . '){Jha.dom.$(\'task\').value = \'' . $tasks[$i] . '\'; document.forms.adminForm.submit();}';
    	}
    	else {
    		$onclick = 'Jha.dom.$(\'task\').value = \'' . $tasks[$i] . '\'; document.forms.adminForm.submit();';
    	}
      if($icons != null && $icons[$i] != null){
      	$tagicon = new Tag('img');
      	$tagicon->setAttribute('src', $icons[$i]);
      	$tagicon->setAttribute('title', $titles[$i]);
      }
      else{
      	$tagicon = new Tag('span');
      	$tagicon -> setAttribute('class', 'icon-' . $linktypes[$i]);
      	$tagicon -> setAttribute('title', $titles[$i]);
      }
      $tagtr -> add( $tagtd = new Tag('td') );
      $tagtd -> add( $taga = new Tag('a') );
      $taga -> setAttribute('href', 'javascript:;');
      $taga -> setAttribute('onclick', 'javascript:' . $onclick);
      $taga -> add( $tagicon );
      $taga -> add( $titles[$i] );
    }
    return $tagtable->html ();
  }
  
  public function tagHTML($tag = 'p', $extras = array()) {
  	jhaimport('jhaley.web.tags');
    $taginput = new Tag ('input');
    foreach ($extras as $index => $value) {
    	$taginput->setAttribute($index, $value);
    }
    return $taginput->html ();
  }
}
?>
\end{lstlisting}


\section{Modelo de Base de Datos}
El modelo de base de datos o modelo Entidad-Relaci\'on, contiene tablas que son propias del motor de renderizado de plantillas, tambi\'en est\'an presentes algunas tablas que son parte de las extensiones (contenido, menu). Escencialmente son extensiones, pero a la vez son una parte fundamental del \textit{i}CMS.

\begin{figure}[h]
\centering
\includegraphics[scale=.6, keepaspectratio=true]{imagenes/12_imagen.png}
\caption{Modelo de Base de Datos. [Elaboraci\'on propia].}
\end{figure}

\clearpage
